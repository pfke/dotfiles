\documentclass[10pt,landscape]{article}
\usepackage{multicol}
\usepackage{calc}
\usepackage{ifthen}
\usepackage[landscape]{geometry}
\usepackage{hyperref}
\usepackage{amssymb}

% To make this come out properly in landscape mode, do one of the following
% 1.
%  pdflatex latexsheet.tex
%
% 2.
%  latex latexsheet.tex
%  dvips -P pdf  -t landscape latexsheet.dvi
%  ps2pdf latexsheet.ps


% If you're reading this, be prepared for confusion.  Making this was
% a learning experience for me, and it shows.  Much of the placement
% was hacked in; if you make it better, let me know...


% 2008-04
% Changed page margin code to use the geometry package. Also added code for
% conditional page margins, depending on paper size. Thanks to Uwe Ziegenhagen
% for the suggestions.

% 2006-08
% Made changes based on suggestions from Gene Cooperman. <gene at ccs.neu.edu>


% To Do:
% \listoffigures \listoftables
% \setcounter{secnumdepth}{0}


% This sets page margins to .5 inch if using letter paper, and to 1cm
% if using A4 paper. (This probably isn't strictly necessary.)
% If using another size paper, use default 1cm margins.
\ifthenelse{\lengthtest { \paperwidth = 11in}}
	{ \geometry{top=.5in,left=.5in,right=.5in,bottom=.5in} }
	{\ifthenelse{ \lengthtest{ \paperwidth = 297mm}}
		{\geometry{top=1cm,left=1cm,right=1cm,bottom=1cm} }
		{\geometry{top=1cm,left=1cm,right=1cm,bottom=1cm} }
	}

% Turn off header and footer
\pagestyle{empty}
 

% Redefine section commands to use less space
\makeatletter
\renewcommand{\section}{\@startsection{section}{1}{0mm}%
                                {-1ex plus -.5ex minus -.2ex}%
                                {0.5ex plus .2ex}%x
                                {\normalfont\bfseries}}
\renewcommand{\subsection}{\@startsection{subsection}{2}{0mm}%
                                {-1explus -.5ex minus -.2ex}%
                                {0.5ex plus .2ex}%
                                {\normalfont\scriptsize\bfseries}}
\renewcommand{\subsubsection}{\@startsection{subsubsection}{3}{0mm}%
                                {-1ex plus -.5ex minus -.2ex}%
                                {1ex plus .2ex}%
                                {\normalfont\scriptsize\bfseries}}
\makeatother

% Define BibTeX command
\def\BibTeX{{\rm B\kern-.05em{\sc i\kern-.025em b}\kern-.08em
    T\kern-.1667em\lower.7ex\hbox{E}\kern-.125emX}}

% Don't print section numbers
\setcounter{secnumdepth}{0}


\setlength{\parindent}{0pt}
\setlength{\parskip}{0pt plus 0.5ex}


% -----------------------------------------------------------------------

\begin{document}

\raggedright
\footnotesize
\begin{multicols}{5}


% multicol parameters
% These lengths are set only within the two main columns
%\setlength{\columnseprule}{0.25pt}
\setlength{\premulticols}{1pt}
\setlength{\postmulticols}{1pt}
\setlength{\multicolsep}{1pt}
\setlength{\columnsep}{2pt}

\begin{center}
     \normalsize{\textbf{nvim Cheat Sheet}} \\
\end{center}

\section{\hrulefill global\hrulefill}
\begin{tabular}{@{}ll@{}}
    \verb!:viusage!     & show all key maps \\
    \verb!:PlugClean!   & uninstall plugs \\
    \verb!:PlugInstall! & install plugs \\
    \verb!:PlugUpdate!  & update plugs \\
    \verb!,0..9!       & goto buf \# \\
\end{tabular}

\section{\hrulefill editing\hrulefill}
\subsection{:: general ::}
\begin{tabular}{@{}ll@{}}
    \verb!go!       & newline below (no imode) \\
    \verb!gO!       & newline above (no imode) \\
\end{tabular}

\subsection{:: multiple cursors ::}
\begin{tabular}{@{}ll@{}}
    \verb!<C-n>!    & next (start selection) \\
    \verb!<C-x>!    & skip \\
    \verb!<C-p>!    & prev \\
\end{tabular}

\section{\hrulefill movement\hrulefill}
\subsection{:: general ::}
\begin{tabular}{@{}ll@{}}
    \verb!h j k l!     & left, down, up, right \\
    \verb!<C-h> <C-l>! & prev/next block \\
    \verb!<C-j> <C-k>! & 20 lines down/up \\
    \verb!0 $!         & begin/end of line \\
    \verb!<C-o> <C-i>! & jump back/fore \\
    \verb!<C-i>!       & 20 lines down/up \\
\end{tabular}

\subsection{:: easymotion (emotion) ::}
\begin{tabular}{@{}ll@{}}
    \verb!sd!     & 1 char \\
    \verb!sf!     & 1 char (overwin) \\
    \verb!sh ,h!  & 1 char (line back) \\
    \verb!sl ,l!  & 1 char (line fore) \\
    \verb!sj ,j!  & 1 char (down) \\
    \verb!sk ,k!  & 1 char (up) \\
    \verb!ss!     & 2 chars \\
    \verb!s/!     & x chars \\
    \verb!sn!     & goto next match \\
\end{tabular}

\section{\hrulefill searching\hrulefill}
\subsection{:: general ::}
\begin{tabular}{@{}ll@{}}
    \verb!<Esc><Esc>! & clear search \\
    \verb!,*!         & highlight word \\
    \verb!<C-r>!      & replace vselected word \\
\end{tabular}

\subsection{:: incsearch ::}
\begin{tabular}{@{}ll@{}}
    \verb!/!            & search forward \\
    \verb!?!            & search backward \\
    \verb!g/!           & search (stay) \\
    \verb!<S-z>/!       & fuzzy search forward \\
    \verb!<S-z>?!       & fuzzy search backward \\
    \verb!<S-z>g/!      & fuzzy search (stay) \\
    \verb!z/!           & emotion search (fore) \\
    \verb!z?!           & emotion search (back) \\
    \verb!zg/!          & emotion search (stay) \\
    \verb!<Tab>!        & move to next match \\
    \verb!<S-Tab>!      & move to prev match \\
    \verb!<C-j>!        & move to next page match \\
    \verb!<C-k>!        & move to prev page match \\
\end{tabular}

\section{\hrulefill selection\hrulefill}
\subsection{:: general ::}
\begin{tabular}{@{}ll@{}}
    \verb!+!            & expand \\
    \verb!-!            & shrink \\
\end{tabular}

\section{\hrulefill NERDTree\hrulefill}
\subsection{:: NERDTree - general ::}
\begin{tabular}{@{}ll@{}}
\verb!,<Tab>!   & toggle NERDtree \\
\verb!,<S-Tab>! & find file in NERDTree \\
\verb!q!        & close window \\
\verb!A!        & min-max window \\
\verb!?!        & toggle help \\
\end{tabular}

\subsection{:: NERDTree - file node ::}
\begin{tabular}{@{}ll@{}}
\verb!o!        & open \\
\verb!go!       & open (preview) \\
\verb!t!        & open in new tab \\
\verb!T!        & open in new tab silently \\
\verb!i!        & open in split \\
\verb!gi!       & open in split (preview) \\
\verb!s!        & open in vsplit \\
\verb!gs!       & open in vsplit (preview) \\
\end{tabular}

\subsection{:: NERDTree - dir node ::}
\begin{tabular}{@{}ll@{}}
\verb!o!        & open+close node \\
\verb!O!        & recursively open node \\
\verb!t!        & open in new tab \\
\verb!T!        & open in new tab silently \\
\verb!x!        & close parent of node \\
\verb!X!        & close all child nodes \\
\verb!s!        & open in vsplit \\
\verb!gs!       & open in vsplit (preview) \\
\end{tabular}

\subsection{:: NERDTree - bookmarks ::}
\begin{tabular}{@{}ll@{}}
\verb!o!        & open bookmark \\
\verb!go!       & preview file \\
\verb!go!       & find file file in tree \\
\verb!t!        & open file in new tab \\
\verb!T!        & open file in new tab silently \\
\verb!D!        & delete bookmark \\
\end{tabular}
\begin{tabular}{@{}ll@{}}
\verb!:Bookmark <...>!       & create \\
\verb!:ClearBookmarks <...>! & delete \\
\verb!:EditBookmark!         & edit \\
\end{tabular}

\subsection{:: NERDTree - tree nav ::}
\begin{tabular}{@{}ll@{}}
\verb!P!        & go to root \\
\verb!p!        & go to parent \\
\verb!K!        & go to first child \\
\verb!J!        & go to last child \\
\verb!<C-j>!    & go to next sibling \\
\verb!<C-k>!    & go to prev sibling \\
\end{tabular}

\subsection{:: NERDTree - filesystem ::}
\begin{tabular}{@{}ll@{}}
\verb!C!        & change tree root to selected dir \\
\verb!u!        & move root up a dir \\
\verb!U!        & move root up a dir (leave root) \\
\verb!r!        & refresh cursor dir \\
\verb!R!        & refresh root \\
\verb!m!        & show menu \\
\verb!cd!       & change the CWD \\
\verb!CD!       & change tree root to CWD \\
\end{tabular}

\subsection{:: NERDTree - filtering ::}
\begin{tabular}{@{}ll@{}}
    \verb!I!    & hidden files \\
    \verb!F!    & show/hide files \\
    \verb!B!    & show/hide bookmarks \\
\end{tabular}


\section{\hrulefill FZF\hrulefill}
\subsection{:: fzf - git ::}
\begin{tabular}{@{}ll@{}}
    \verb!,fgb!     & blame \\
    \verb!,fgc!     & commits \\
    \verb!,fgd!     & diff \\
    \verb!,fgf!     & files \\
    \verb!,fgs!     & status \\
    \verb!,fgl!     & log \\
\end{tabular}

\subsection{:: fzf - search ::}
\begin{tabular}{@{}ll@{}}
    \verb!,fa!      & ag \\
    \verb!,fb!      & buffers \\
    \verb!,fü!      & color schemes \\
    \verb!,fc!      & command history \\
    \verb!,ff!      & files \\
    \verb!,fl!      & lines \\
    \verb!,fm!      & maps \\
    \verb!,fr!      & marks \\
    \verb!,fs!      & snippets \\
    \verb!,fo!      & old files+buffers \\
    \verb!,f/!      & search history \\
\end{tabular}

\section{\hrulefill IDE\hrulefill}
\subsection{:: neomake ::}
\begin{tabular}{@{}ll@{}}
    \verb!<F3>!     & prev err/warn \\
    \verb!<F4>!     & next err/warn \\
\end{tabular}

\section{\hrulefill vcs\hrulefill}
\subsection{:: signify ::}
\begin{tabular}{@{}ll@{}}
    \verb!,n!       & next hunk \\
    \verb!,<S-n>!   & prev hunk \\
\end{tabular}





\rule{0.3\linewidth}{0.25pt}
\scriptsize

Copyright \copyright\ 2019 Heiko BLobner

\end{multicols}
\end{document}

