\documentclass[defaultpackages]{vim-cheatsheet}

\usepackage{lipsum}
\usepackage{graphicx}
\usepackage[T1]{fontenc}
\usepackage{pifont}

\title{nvim cheatsheet}
\author{Heiko Blobner}

\begin{document}

\tableofcontents

\section{\hrulefill general\hrulefill}

\subsection{key modifier}
\begin{tabular}{@{}ll@{}}
    \verb!M- A-!        & ALT key \\
    \verb!C-!           & CTRL key \\
\end{tabular}

\subsection{commands}
\begin{tabular}{@{}ll@{}}
    \verb!:viusage!         & show all key maps \\
    \verb!:PlugClean!       & uninstall plugs \\
    \verb!:PlugInstall!     & install plugs \\
    \verb!:PlugUpdate!      & update plugs \\
    \verb!,0..9!            & goto buf \# \\
    \verb!,<Space>!         & distraction free view \\
    \verb!:map!             & show mapped keys \\
    \verb!:map!!            & show !mapped keys \\
    \verb!:h key-notation!  & help for ... \\
    \verb!:reg!             & displays reg content \\
\end{tabular}

\subsection{help}
\begin{tabular}{@{}ll@{}}
    \verb!:h!           & show help \\
    \verb!<C-]!         & follow link \\
    \verb!<C-T!         & back to prev topic \\
\end{tabular}


\section{\hrulefill motion\hrulefill}

\subsection{motions and operators}

\subsection{left/right motions}

\subsection{up/down motions}

\subsection{text object motion}

\subsection{text object selection}
This is a series of commands that can only be used while in Visual mode or
after an operator.  The commands that start with "a" select "a"n object
including white space, the commands starting with "i" select an "inner" object
without white space, or just the white space.  Thus the "inner" commands
always select less text than the "a" commands. \\

\begin{tabular}{@{}ll@{}}
    \verb!aw!       & "a word", select [count] words \\
    \verb!iw!       & "inner word", select [count] words \\
    \verb!aW!       & "a WORD", select [count] WORDS \\
    \verb!iW!       & "inner WORD", select [count] WORDS \\
    \verb!as!       & "a sentence", select [count] sentences \\
    \verb!is!       & "inner sentence", select [count] sentences \\
    \verb!ap!       & "a paragraph", select [count] paragraphs \\
    \verb!ip!       & "inner paragraph", select [count] paragraphs \\
    \verb!a] a[!    & "a [] block", select [count] '[' ']' blocks. \\
    \verb!i] i[!    & "inner [] block", select [count] '[' ']' blocks. \\
    \verb!a) a( ab! & "a block", select [count] blocks, from "[count] [(" to the matching ')', including the '(' and ')' \\
    \verb!i) i( ib! & "inner block", select [count] blocks, from "[count] [( to the matching ')', excluding the '(' and ')' \\
    \verb!a> a<!    & "a <> block", select [count] <> blocks, from the [count]'th unmatched '<' backwards to the matching '>', including the '<' and '>'. \\
    \verb!i> i<!    & "inner <> block", select [count] <> blocks, from the [count]'th unmatched '<' backwards to the matching '>', excluding the '<' and '>'. \\
    \verb!at!       & "a tag block", select [count] tag blocks \\
%%			[count]'th unmatched "<aaa>" backwards to the matching
%%			"</aaa>", including the "<aaa>" and "</aaa>".
%%			See |tag-blocks| about the details.
%%			When used in Visual mode it is made characterwise.
%%
%%						*v_it* *it*
%    \verb!%it			"inner tag block", select [count] tag blocks, from the
%%			[count]'th unmatched "<aaa>" backwards to the matching
%%			"</aaa>", excluding the "<aaa>" and "</aaa>".
%%			See |tag-blocks| about the details.
%%			When used in Visual mode it is made characterwise.
%%
%    \verb!%a}							*v_a}* *a}* *a{*
%    \verb!%a{							*v_aB* *v_a{* *aB*
%    \verb!%aB			"a Block", select [count] Blocks, from "[count] [{" to
%%			the matching '}', including the '{' and '}' (see
%%			|[{|).
%%			When used in Visual mode it is made characterwise.
%%
%    \verb!%i}							*v_i}* *i}* *i{*
%    \verb!%i{							*v_iB* *v_i{* *iB*
%    \verb!%iB			"inner Block", select [count] Blocks, from "[count] [{"
%%			to the matching '}', excluding the '{' and '}' (see
%%			|[{|).
%%			When used in Visual mode it is made characterwise.
%%
%    \verb!%a"							*v_aquote* *aquote*
%    \verb!%a'							*v_a'* *a'*
%    \verb!%a`							*v_a`* *a`*
%%			"a quoted string".  Selects the text from the previous
%%			quote until the next quote.  The 'quoteescape' option
%%			is used to skip escaped quotes.
%%			Only works within one line.
%%			When the cursor starts on a quote, Vim will figure out
%%			which quote pairs form a string by searching from the
%%			start of the line.
%%			Any trailing white space is included, unless there is
%%			none, then leading white space is included.
%%			When used in Visual mode it is made characterwise.
%%			Repeating this object in Visual mode another string is
%%			included.  A count is currently not used.
%%
%    \verb!%i"							*v_iquote* *iquote*
%    \verb!%i'							*v_i'* *i'*
%    \verb!%i`							*v_i`* *i`*
%%			Like a", a' and a`, but exclude the quotes and
%%			repeating won't extend the Visual selection.
%%			Special case: With a count of 2 the quotes are
%%			included, but no extra white space as with a"/a'/a`.
%%
%%When used after an operator:
%%For non-block objects:
%%	For the "a" commands: The operator applies to the object and the white
%%	space after the object.  If there is no white space after the object
%%	or when the cursor was in the white space before the object, the white
%%	space before the object is included.
%%	For the "inner" commands: If the cursor was on the object, the
%%	operator applies to the object.  If the cursor was on white space, the
%%	operator applies to the white space.
%%For a block object:
%%	The operator applies to the block where the cursor is in, or the block
%%	on which the cursor is on one of the braces.  For the "inner" commands
%%	the surrounding braces are excluded.  For the "a" commands, the braces
%%	are included.
%%
%%When used in Visual mode:
%%When start and end of the Visual area are the same (just after typing "v"):
%%	One object is selected, the same as for using an operator.
%%When start and end of the Visual area are not the same:
%%	For non-block objects the area is extended by one object or the white
%%	space up to the next object, or both for the "a" objects.  The
%%	direction in which this happens depends on which side of the Visual
%%	area the cursor is.  For the block objects the block is extended one
%%	level outwards.



    \verb!dl!       & delete character (alias: "x") \\
    \verb!diw!      & delete inner word \\
    \verb!daw!      & delete a word \\
    \verb!diW!      & delete inner WORD (see WORD) \\
    \verb!daW!      & delete a WORD (see WORD) \\
    \verb!dgn!      & delete the next search pattern match \\
    \verb!dd!       & delete one line \\
    \verb!dis!      & delete inner sentence \\
    \verb!das!      & delete a sentence \\
    \verb!dib!      & delete inner '(' ')' block \\
    \verb!dab!      & delete a '(' ')' block \\
    \verb!dip!      & delete inner paragraph \\
    \verb!dap!      & delete a paragraph \\
    \verb!diB!      & delete inner '{' '}' block \\
    \verb!daB!      & delete a '{' '}' block \\
\end{tabular}

Also see `gn` and `gN`, operating on the last search pattern.

\section{\hrulefill editing\hrulefill}
\subsection{:: general ::}
\begin{tabular}{@{}ll@{}}
    \verb!c d y!    & change, delete, yank \\
    \verb!s!        & replace \\
    \verb!v!        & visual select \\
    \verb!x!        & delect char \\
    \verb!go!       & newline below (no imode) \\
    \verb!gO!       & newline above (no imode) \\
\end{tabular}

\subsection{:: basic: delete ::}
\begin{tabular}{@{}ll@{}}
    \verb!dl!       & delete character (alias: "x") \\
    \verb!diw!      & delete inner word \\
    \verb!daw!      & delete a word \\
    \verb!diW!      & delete inner WORD (see WORD) \\
    \verb!daW!      & delete a WORD (see WORD) \\
    \verb!dgn!      & delete the next search pattern match \\
    \verb!dd!       & delete one line \\
    \verb!dis!      & delete inner sentence \\
    \verb!das!      & delete a sentence \\
    \verb!dib!      & delete inner '(' ')' block \\
    \verb!dab!      & delete a '(' ')' block \\
    \verb!dip!      & delete inner paragraph \\
    \verb!dap!      & delete a paragraph \\
    \verb!diB!      & delete inner '{' '}' block \\
    \verb!daB!      & delete a '{' '}' block \\
\end{tabular}

\subsection{:: autopairs ::}
\begin{tabular}{@{}ll@{}}
    \verb!<C-v>!    & insert pair w/o plugin \\
    \verb!<A-p>!    & disable/enable plugin \\
    \verb!<A-e>!    & \{|\}'hello' \ding{217} \{'hello'\} \\
\end{tabular}

\subsection{:: surround.vim ::}

\subsubsection{:: targets ::}
\begin{tabular}{@{}ll@{}}
    \verb!w W s! &  word, WORD, sentence \\
\end{tabular}

\subsubsection{:: commands ::}
\begin{tabular}{@{}ll@{}}
    \verb!ds !      & delete surroundings \\
    \verb!  ds"!    & \hspace{2pt} "hell*o" \ding{217} hello \\
    \verb!  ds)!    & \hspace{2pt} (1+5*6)/2 \ding{217} 1+56/2 \\
    \verb!  dst!    & \hspace{2pt} <p>*h</p> \ding{217} h \\
    \verb!cs !      & change surroundings \\
    \verb!  cs"'  ! & \hspace{2pt} "H *w!" \ding{217} 'H w!' \\
    \verb!  cs"<q>! & \hspace{2pt} "H *w!" \ding{217} <q>H w!</q> \\
    \verb!  cs)]  ! & \hspace{2pt} (1+5*6)/2 \ding{217} [1+56]/2 \\
    \verb!  cs)[  ! & \hspace{2pt} (1+5*6)/2 \ding{217} [ 1+56 ]/2 \\
    \verb!  cst<p>! & \hspace{2pt} <a>.*</a> \ding{217} <p>.</p> \\
    \verb!ys !      & you surroundings \\
    \verb!  ysiw) ! & \hspace{2pt} Hee *wor! \ding{217} 'Hee (wor)! \\
    \verb!  ysiW) ! & \hspace{2pt} Hee *wor! \ding{217} 'Hee (wor!) \\
    \verb!ysW(!     & add: if *x>3 ( \ding{217} if ( x>3 ) ( \\
    \verb!cs])!     & chg: [123+5*6]/2 \ding{217} (123+56)/2 \\
    \verb!vllllS'!  & visual: my str = *wheel; -\ding{217} my str = 'wheel'; \\
\end{tabular}

(isumsi wwww)

\subsection{:: multiple cursors ::}
\begin{tabular}{@{}ll@{}}
    \verb!<C-n>!    & start/next key \\
    \verb!g<C-n>!   & start/next key (w/o bounds) \\
    \verb!<C-n>!    & next key \\
    \verb!<C-p>!    & prev key \\
    \verb!<C-x>!    & skip key \\
    \verb!<A-n>!    & select all keys \\
    \verb!g<A-n>!   & select all keys (w/o bounds) \\
\end{tabular}


\section{\hrulefill movement\hrulefill}

\subsection{:: general ::}
\begin{tabular}{@{}ll@{}}
    \verb!h j k l!          & left, down, up, right \\
    \verb!<C-h> <C-l>!      & prev/next empty line \\
    \verb!<C-j> <C-k>!      & 20 lines down/up \\
    \verb!<C-o> <C-i>!      & jump back/forward \\
    \verb!0 $!              & begin/end of line \\
    \verb!gg G!             & begin/end of file \\
    \verb!%!                & match block \\
\end{tabular}

\subsection{:: visual split ::}
Splitting on visual selection. \\
\begin{tabular}{@{}ll@{}}
    \verb!<C-w>gr!          & resize split to selection \\
    \verb!<C-w>gss!         & split out vsel \\
    \verb!<C-w>gsa!         & split out vsel above \\
    \verb!<C-w>gsb!         & split out vsel below \\
\end{tabular}

\subsection{:: easymotion (emotion) ::}
\begin{tabular}{@{}ll@{}}
    \verb!sd!     & 1 char \\
    \verb!sf!     & 1 char (overwin) \\
    \verb!sh ,h!  & 1 char (line back) \\
    \verb!sl ,l!  & 1 char (line fore) \\
    \verb!sj ,j!  & 1 char (down) \\
    \verb!sk ,k!  & 1 char (up) \\
    \verb!ss!     & 2 chars \\
    \verb!s/!     & x chars \\
    \verb!sn!     & goto next match \\
\end{tabular}

\section{\hrulefill split screen\hrulefill}

\subsection{:: general ::}
\begin{tabular}{@{}ll@{}}
    \verb!<C-w>v!           & new vertical split \\
    \verb!<C-w>s!           & new horizontal split \\
    \verb!<C-w>c!           & close split \\
    \verb!<C-w>o!           & close other splits \\
    \verb!<C-w>H!           & close split \\
\end{tabular}

\subsection{:: moving cursor ::}
\begin{tabular}{@{}ll@{}}
    \verb!<A-h> <A-l>!      & navigate left/right \\
    \verb!<A-j> <A-k>!      & navigate up/down \\
\end{tabular}

\subsection{:: moving windows ::}
\begin{tabular}{@{}ll@{}}
    \verb!<C-w>r!           & rotate windows \\
    \verb!<C-w>x!           & swap windows \\
    \verb!<C-w>K!           & move top; full width \\
    \verb!<C-w>J!           & move bottom; full width \\
    \verb!<C-w>H!           & move left; full height \\
    \verb!<C-w>L!           & move right; full height \\
    \verb!<C-w>T!           & move to new tab \\
\end{tabular}

\subsection{:: resizing ::}
\begin{tabular}{@{}ll@{}}
    \verb!<C-w>=!               & make equally h/w \\
    \verb!<C-w>+!               & inc height \\
    \verb!<C-w>-!               & dec height \\
    \verb!<C-PageUp>!           & inc height \\
    \verb!<C-PageDown>!         & dec height \\
    \verb!<C-w>>!               & inc width \\
    \verb!<C-w><!               & dec width \\
    \verb!:res[ize] N!          & set height to N \\
    \verb!:res[ize] +N!         & grow height by N \\
    \verb!:res[ize] -N!         & shrink height by N \\
    \verb!:vertical res N!      & set width to N \\
    \verb!:vertical res +N!     & grow width by N \\
    \verb!:vertical res -N!     & shrink width by N \\
\end{tabular}


\section{\hrulefill searching\hrulefill}

\subsection{:: plugins ::}
\begin{tabular}{@{}ll@{}}
    \verb!anzu!         & cmd line status \\
    \verb!asterisk!     & improved * (visual mode) \\
    \verb!incsearch!    & search itself \\
\end{tabular}

\subsection{:: general ::}
\begin{tabular}{@{}ll@{}}
    \verb!<Esc><Esc>!   & clear search \\
    \verb!<C-r>!        & replace vselected word \\
\end{tabular}

\subsection{:: incsearch ::}
\begin{tabular}{@{}ll@{}}
    \verb!/!            & search forward \\
    \verb!?!            & search backward \\
    \verb!g/!           & search (stay) \\
    \verb!n N!          & next/prev match \\
    \verb!<S-z>/!       & fuzzy search forward \\
    \verb!<S-z>?!       & fuzzy search backward \\
    \verb!<S-z>g/!      & fuzzy search (stay) \\
    \verb!z/!           & emotion search (fore) \\
    \verb!z?!           & emotion search (back) \\
    \verb!zg/!          & emotion search (stay) \\
    \verb!<Tab>!        & move to next match \\
    \verb!<S-Tab>!      & move to prev match \\
    \verb!<C-j>!        & move to next page match \\
    \verb!<C-k>!        & move to prev page match \\
\end{tabular}

\subsection{:: asterisk ::}
\begin{tabular}{@{}ll@{}}
    \verb!*!            & match word (forward; stay) \\
    \verb!#!            & match word (backward) \\
    \verb!g*!           & search word (forward; stay) \\
    \verb!g#!           & search word (backward) \\
\end{tabular}



\section{\hrulefill NERDTree\hrulefill}
\subsection{:: NERDTree - general ::}
\begin{tabular}{@{}ll@{}}
\verb!,<Tab>!   & toggle NERDtree \\
\verb!,<S-Tab>! & find file in NERDTree \\
\verb!q!        & close window \\
\verb!A!        & min-max window \\
\verb!?!        & toggle help \\
\end{tabular}

\subsection{:: NERDTree - file node ::}
\begin{tabular}{@{}ll@{}}
\verb!o!        & open \\
\verb!go!       & open (preview) \\
\verb!t!        & open in new tab \\
\verb!T!        & open in new tab silently \\
\verb!i!        & open in split \\
\verb!gi!       & open in split (preview) \\
\verb!s!        & open in vsplit \\
\verb!gs!       & open in vsplit (preview) \\
\end{tabular}

\subsection{:: NERDTree - dir node ::}
\begin{tabular}{@{}ll@{}}
\verb!o!        & open+close node \\
\verb!O!        & recursively open node \\
\verb!t!        & open in new tab \\
\verb!T!        & open in new tab silently \\
\verb!x!        & close parent of node \\
\verb!X!        & close all child nodes \\
\verb!s!        & open in vsplit \\
\verb!gs!       & open in vsplit (preview) \\
\end{tabular}

\subsection{:: NERDTree - bookmarks ::}
\begin{tabular}{@{}ll@{}}
\verb!o!        & open bookmark \\
\verb!go!       & preview file \\
\verb!go!       & find file file in tree \\
\verb!t!        & open file in new tab \\
\verb!T!        & open file in new tab silently \\
\verb!D!        & delete bookmark \\
\end{tabular}
\begin{tabular}{@{}ll@{}}
\verb!:Bookmark <...>!       & create \\
\verb!:ClearBookmarks <...>! & delete \\
\verb!:EditBookmark!         & edit \\
\end{tabular}

\subsection{:: NERDTree - tree nav ::}
\begin{tabular}{@{}ll@{}}
\verb!P!        & go to root \\
\verb!p!        & go to parent \\
\verb!K!        & go to first child \\
\verb!J!        & go to last child \\
\verb!<C-j>!    & go to next sibling \\
\verb!<C-k>!    & go to prev sibling \\
\end{tabular}

\subsection{:: NERDTree - filesystem ::}
\begin{tabular}{@{}ll@{}}
\verb!C!        & change tree root to selected dir \\
\verb!u!        & move root up a dir \\
\verb!U!        & move root up a dir (leave root) \\
\verb!r!        & refresh cursor dir \\
\verb!R!        & refresh root \\
\verb!m!        & show menu \\
\verb!cd!       & change the CWD \\
\verb!CD!       & change tree root to CWD \\
\end{tabular}

\subsection{:: NERDTree - filtering ::}
\begin{tabular}{@{}ll@{}}
    \verb!I!    & hidden files \\
    \verb!F!    & show/hide files \\
    \verb!B!    & show/hide bookmarks \\
\end{tabular}


\section{\hrulefill FZF\hrulefill}
\subsection{:: fzf - git ::}
\begin{tabular}{@{}ll@{}}
    \verb!,fgb!     & blame \\
    \verb!,fgc!     & commits \\
    \verb!,fgd!     & diff \\
    \verb!,fgf!     & files \\
    \verb!,fgs!     & status \\
    \verb!,fgl!     & log \\
\end{tabular}

\subsection{:: fzf - search ::}
\begin{tabular}{@{}ll@{}}
    \verb!,fa!      & ag \\
    \verb!,fb!      & buffers \\
    \verb!,fü!      & color schemes \\
    \verb!,fc!      & command history \\
    \verb!,ff!      & files \\
    \verb!,fl!      & lines \\
    \verb!,fm!      & maps \\
    \verb!,fr!      & marks \\
    \verb!,fs!      & snippets \\
    \verb!,fo!      & old files+buffers \\
    \verb!,f/!      & search history \\
\end{tabular}


\section{\hrulefill IDE\hrulefill}
\subsection{:: neomake ::}
\begin{tabular}{@{}ll@{}}
    \verb!<F3>!     & prev err/warn \\
    \verb!<F4>!     & next err/warn \\
\end{tabular}

\subsection{:: deoplete ::}
\begin{tabular}{@{}ll@{}}
    \verb!<C-P> <C-N!   & up/down the completion popup \\
    \verb!<F4>!     & next err/warn \\
\end{tabular}

\section{\hrulefill vcs\hrulefill}
\subsection{:: signify ::}
\begin{tabular}{@{}ll@{}}
    \verb!,n!       & next hunk \\
    \verb!,<S-n>!   & prev hunk \\
\end{tabular}

%\section{Det første som skjer}
%
%\lipsum[1-1]
%
%\begin{equation}
%	\sum_{n=1}^\infty 10^{-n!}=0.110001\cdots\label{eq_1}
%\end{equation}
%
%As you can see in Equation~\ref{eq_1}, it is a transcendental number, $3 + 5 = 8$.
%
%\lipsum[2-2]
%
%\[ \int_0^\infty f(x)\, dx = e^x \]
%
%\lipsum[1-1]
%
%\subsection{Enda en test}
%
%\begin{figure}[H]
%\begin{center}
%%\includegraphics[width=50px] {crab_nebula.jpg}
%\end{center}
%\caption{A binary search tree, but this caption is a lot longer. Trying to make it a lot of lines to test spacing in the captions.}\label{fig:ackseq}
%\end{figure}
%
%\lipsum[1-1]
%
%\section*{Unnumbered section}
%
%\subsubsection{Lorem ipsum dolor sit amet}
%
%\lipsum[1-1]
%
%\section{kake}
%
%\lipsum[3-50]

\end{document}
