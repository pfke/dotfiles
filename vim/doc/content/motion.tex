\section{\hrulefill motion\hrulefill}

\subsection{motions and operators}
\begin{tabular}{@{}ll@{}}
    \verb!c!                & change \\
    \verb!d!                & delete \\
    \verb!y!                & yank into register \\
    \verb!~!                & swap case \\
    \verb!g~~!              & swap case (entire ll.) \\
    \verb!gu gU!            & make lower-/uppercase \\
    \verb!> <!              & shift right/left \\
\end{tabular}

\subsection{left-right motions}
\begin{tabular}{@{}ll@{}}
    \verb!h l!              & [cnt] left, right \\
    \verb!0 $!              & begin/end of ll. \\
    \verb!^!                & 1st non-blank of ll. \\
    \verb!g_!               & last non-blank of ll. \\

    \verb!g0 gm g$!         & beg,mid,end of scrn ll. \\

    \verb!|!                & to scrn. col [cnt] \\

    \verb!F(_) f(_)!        & to [cnt]'th (char) (\rotatebox[origin=c]{180}{\ding{217}}/\ding{217}) \\
    \verb!T(_) t(_)!        & till [cnt]'th (char) (\rotatebox[origin=c]{180}{\ding{217}}/\ding{217}) \\
    \verb!; ,!              & [cnt] rpt last F,f,T,t (\rotatebox[origin=c]{180}{\ding{217}}/\ding{217}) \\
\end{tabular}


\subsection{up-down motions}
\begin{tabular}{@{}ll@{}}
    \verb!j k!              & [cnt] down, up \\
    \verb!gj gk!            & [cnt] disp ll. down, up \\
    \verb!- +!              & [cnt] ll. up/down 1st chr \\
    \verb!<C-j> <C-k>!      & 20 ll. down/up \\

    \verb!<C-h> <C-l>!      & prev/next empty ll. \\

    \verb!gg G!             & goto ll. [cnt] def. 1st/last \\
    \verb!%!                & goto [count] perc of file \\
\end{tabular}


\subsection{word motion}
\begin{tabular}{@{}ll@{}}
    \verb!w W!      & word/WORD (\ding{217}) \\
    \verb!e E!      & end of word/WORD (\ding{217}) \\
    \verb!b B!      & word/WORD (\rotatebox[origin=c]{180}{\ding{217}}) \\
    \verb!ge gE!    & end of word/WORD (\rotatebox[origin=c]{180}{\ding{217}}) \\
\end{tabular}

\hrulefill \noindent\rule{4cm}{0.4pt} \\
A \textbf{word} consists of a sequence of letters, digits and underscores, or a sequence of other non-blank characters, separated with white space (spaces, tabs, <EOL>). \\
A \textbf{WORD} consists of a sequence of non-blank characters, separated with white space. An empty line is also considered to be a WORD. \\


\subsection{text object motion}
\begin{tabular}{@{}ll@{}}
    \verb!( )!      & [count] sentences (\rotatebox[origin=c]{180}{\ding{217}}/\ding{217}) \\
    \verb!{ }!      & [count] paragraphs (\rotatebox[origin=c]{180}{\ding{217}}/\ding{217}) \\
    \verb!]] ][!    & [count] to the next '\{'/'\}' \\
    \verb![[ []!    & [count] to the prev '\{'/'\}' \\
\end{tabular}

\hrulefill \noindent\rule{4cm}{0.4pt} \\
A \textbf{sentence} is defined as ending at a '.', '!' or '?' followed by either the end of a line, or by a space or tab. \\
A \textbf{paragraph} begins after each empty line. \\
A \textbf{section} begins after a form-feed (<C-L>) in the first column. \\


\subsection{text object selection}
This cmds can only be used while in Visual mode or after an operator.  The "a" cmds select an object including white space, the "i" cmds select an "inner" object without white space. \\
\begin{tabular}{@{}ll@{}}
    \verb!aw aW!    & a word/WORD \\
    \verb!iw iW!    & inner word/WORD \\
    \verb!as is!    & a/inner sentence \\
    \verb!ap ip!    & a/inner paragraph \\
    \verb!a] i]!    & a/inner [] block \\
    \verb!a) ab!    & a () block \\
    \verb!i) ib!    & inner () block \\
    \verb!a> i>!    & a/inner <> block \\
    \verb!at it!    & a/inner <a>...</a> block \\
    \verb!a} aB!    & a {} block \\
    \verb!i} iB!    & inner {} block \\
    \verb!a" a' a`! & a quoted string \\
    \verb!i" i' i`! & inner quoted string \\
\end{tabular}

\hrulefill \noindent\rule{4cm}{0.4pt} \\
Examples: \\
\begin{tabular}{@{}ll@{}}
    \verb!dl!       & del char \\
    \verb!daw diw!  & del a/inner word \\
    \verb!dgn dgN!  & del the next/prev match \\
    \verb!dd!       & del [cnt] line \\
    \verb!das!      & del a sentence \\
    \verb!dib!      & del inner () block \\
\end{tabular}


\subsection{marks}
Jumping: \\
\begin{tabular}{@{}ll@{}}
    \verb!` (backtick)!         & specified location \\
    \verb!' (single quote)!     & ll. (1st non-blank) \\
\end{tabular}

\hrulefill \noindent\rule{4cm}{0.4pt} \\
Marks: \\
\begin{tabular}{@{}ll@{}}
    \verb!'{a-z}!       & valid within one file \\
    \verb!'{A-Z}!       & valid between files \\
    \verb!'[ `[!        & prev chged/yanked (1st char) \\
    \verb!'] `]!        & prev chged/yanked (last char) \\
    \verb!'< `>!        & 1st/last ll. visual sel \\
    \verb!'' ``!        & before latest jmp \\
    \verb!'^ `^!        & where Insert mode exited \\
    \verb!'. `.!        & where last chg were made \\

    \verb!]' ]`!        & next ll. w/ lowercase mark \\
    \verb![' [`!        & prev ll. w/ lowercase mark \\
\end{tabular}

\hrulefill \noindent\rule{4cm}{0.4pt} \\
\begin{tabular}{@{}ll@{}}
    \verb!m{a-zA-Z}!            & set mark {a-zA-Z} at cursor \\
    \verb!m' m`!                & set ctxt mark; jmp \verb!'' ``! \\
    \verb!m[ m]!                & set \verb!'[ ']! mark \\
    \verb!m< m>!                & set \verb!'< '>! mark; chg gv \\

    \verb!'{a-z}!               & jump (current buf) \\
    \verb!'{A-Z}!               & jump (chg file) \\

    \verb!:marks!               & list marks \\
    \verb!:delm {_}!            & del marks \\
    \verb!:delm!!               & del current buf marks \\
\end{tabular}


\subsection{jumps}
\begin{tabular}{@{}ll@{}}
    \verb!<C-o> <C-i>!      & jump back/forward \\
    \verb!:ju[mps]!         & print jump list \\
    \verb!:cle[arjumps]!    & clear jump list \\
\end{tabular}

